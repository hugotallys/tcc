\chapter{Introdução}\label{cap:introduction}

A redundância cinemática na robótica, refere-se ao uso de graus liberdade (DOF) adicionais 
além do mínimo necessário para uma executar determinada tarefa. No contexto de robôs industriais e manipuladores,
é importante abordar e resolver problemas relacionados às \emph{singularidades cinemáticas}, que ocorrem quando a 
configuração de um robô leva à perda de um ou mais DOFs, ocasionando a redução da destreza do manipulador. Em relação 
à cinemática inversa diferencial de manipuladores, as singularidades resultam em comportamento indefinido ou imprevisível do robô,
tendo em vista o número infinito de configurações que existem para uma dada pose do efetuador final.

A importância da redundância robótica na solução de singularidades reside na sua capacidade de fornecer caminhos alternativos
para o planejamento de movimentos. Ao ter DOFs extras, um robô pode manobrar ao redor de configurações singulares, mantendo a
eficiência operacional e a segurança. Essa flexibilidade garante que o robô possa continuar executando suas tarefas mesmo ao
se aproximar ou operar perto de pontos singulares, que são tipicamente inevitáveis em tarefas complexas. Além disso, 
a resolução redundância permite a otimização de outros critérios de desempenho, como minimizar o consumo de energia, reduzir o desgaste, melhorar a
precisão e aumentar as capacidades de evasão de obstáculos.

Recentemente, \cite{li2023pseudo} propuseram um esquema baseado em pseudo-inversa para controle de movimento em nível de aceleração,
demonstrando a viabilidade do método através de simulações com o manipulador robótico PA10. Por outro lado, \cite{kuri2023som}
abordaram a resolução de redundância através da combinação de Mapas Auto-Organizáveis (SOM) e uma matriz de Norma Mínima Ponderada
(WLN), mostrando melhorias significativas no controle de trajetória espacial para um manipulador robótico de 5 graus de liberdade.

\cite{hammond2011configuring} explorou o aumento da resolução de movimento eficaz através da configuração de manipuladores cinemáticos
redundantes, onde a redundância cinemática é usada para otimizar a resolução de movimento para tarefas específicas, destacando o potencial dessas
abordagens em aplicações de micromanipulação e montagem de precisão.

Este trabalho estuda a aplicação da redundância cinemática no algoritmo de controle de movimento de um manipulador serial de 5 DOF.
Implementamos estratégias de controle de movimento baseadas na manipulabilidade usando dois índices: a distância aos limites mecânicos
das juntas e a medida de manipulabilidade de Yoshikawa. Em seguida, propomos uma arquitetura de comunicação robótica que suporta tanto
simulação quanto comandos para robôs reais para implementar as técnicas. Finalmente, apresentamos gráficos e resultados que avaliam o
desempenho dos cenários experimentais enquanto alteramos os parâmetros do controlador.

\section{Motivação}\label{sec:motivation}

%%%% WARN CHATGPT COPY

A abordagem tradicional no projeto de manipuladores teve como foco principal a minimização de custos e manutenção,
utilizando o número mínimo de juntas necessárias para uma executar uma determinada tarefa levando, por exemplo, ao 
desenvolvimento dos robôs \emph{Selective Compliance Assembly Robot Arm} (SCARA) para operações de \emph{pick-and-place}.
No entanto, essa abordagem  minimalista apresenta uma série de limitações em aplicações do mundo real, onde fatores como 
limites de juntas, singularidades e obstáculos no espaço de trabalho estão presentes. A introdução de redundância cinemática
permite o desenvolvimento de técnicas de controle que tornem o manipulador capaz de contornar essas limitações, garantindo
um desempenho robusto e confiável. Ao ter mais graus de liberdade (DOF) do que o estritamente necessário, os manipuladores 
redundantes podem alcançar maior destreza e versatilidade, tornando-os mais adequados para ambientes complexos e 
dinamicamente mutáveis.

A maior destreza proveniente da redundância pode ser observada numa maior flexibilidade para executar uma mesma tarefa de 
diferentes maneiras, como por exemplo evitar colisões com obstáculos ou se afastar dos limites de operação das juntas e singulares
utilizando para isso movimentos internos que não alteram a pose do efetuador final. Além disso, a redundância permite a otimização 
de métricas de desempenho não necessariamente cinemáticas, como a minimização do torque ou do consumo de energia, melhorando a eficiência do sistema.
Vale ressaltar que projetar manipuladores com juntas adicionais e garantir sua confiabilidade operacional é um processo complexo e custoso. Esquemas eficazes 
de resolução de redundância são críticos para o sucesso do planejamento e controle de movimentos,  especialmente em ambientes dinâmicos. Apesar desses desafios, 
os benefícios da redundância  cinemática em aumentar a destreza, versatilidade e eficiência fazem dela uma abordagem interessante em sistemas robóticos avançados.

\section{Objetivos}\label{sec:objectives}

\subsection{Objetivos Gerais}

Obter a modelagem de cinemática inversa diferencial em manipuladores seriais redundantes, possibilitando a execução de
trajetórias no espaço de trabalho que levam em conta não só as restrições cinemáticas do movimento mas também critérios
de desempenho do manipulador.

\subsection{Objetivos Específicos}
\begin{itemize}
	\item Estudar a modelagem de cadeias cinemáticas em manipuladores robóticos seriais
	\item Investigar o uso da cinemática direta e inversa diferenciais para execução de trajetórias cartesianas
	\item Implementar a lei de controle utilizando a cinemática direta diferencial e esquemas de resolução de redundância utilizando a pseudo inversa da matriz jacobiana
	\item Propor um ambiente simulado para execução de experimentos utilizando o modelo virtual do manipulador robótico com 5 graus de liberdade
	\item Analisar o desempenho do esquema de controle em diferentes cenários de execução de trajetórias e critérios de desempenho.
\end{itemize}

\section{Estrutura do texto}\label{sec:structure}

Este trabalho está estruturado de modo a introduzir os conceitos já difundidos da cinemática diferencial em manipuladores redundantes,
bem como apresentar uma abordagem prática para resolução de redundância de um manipulador planar simples, com cinco graus de liberdade.
No capítulo 2, é apresentada a fundamentação teórica, abordando conceitos essenciais como a representação de poses no
espaço, transformações homogêneas, cinemática direta, cinemática diferencial, e a resolução de redundância.
O capítulo 3 descreve a implementação prática, detalhando a simulação de manipuladores robóticos, a arquitetura de comunicação proposta
e o algoritmo \emph{Resolved Rate Motion Control}. No capítulo 4, são apresentados os resultados experimentais divididos em dois cenários
distintos, seguidos pela conclusão no capítulo 5, onde são discutidos os principais pontos abordados e sugestões para trabalhos futuros.
