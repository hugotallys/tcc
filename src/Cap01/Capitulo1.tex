\chapter{Introdução}\label{cap:introduction}

A redundância cinemática na robótica refere-se ao uso de graus de liberdade (DOF) adicionais além do mínimo necessário para executar uma
tarefa específica. Em robôs industriais e manipuladores, singularidades cinemáticas ocorrem quando a configuração do robô resulta na perda
de um ou mais DOFs, reduzindo sua capacidade de movimento. De maneira mais precisa, isso significa que há alguma direção (ou subespaço) 
no espaço cartesiano ao longo da qual é impossível mover o efetuador final, independentemente das velocidades empenhadas nas 
juntas~\cite{craig2004}. 

A importância da redundância está na capacidade de fornecer caminhos alternativos para o planejamento de movimentos,
permitindo que o robô manobre ao redor de configurações singulares, mantendo a eficiência operacional e a segurança. Essa flexibilidade 
assegura que o robô continue a executar suas tarefas mesmo próximo a pontos singulares, comuns em tarefas complexas. Além disso, a resolução
de redundância possibilita a otimização de outros critérios de desempenho, como minimizar o consumo de energia, reduzir o desgaste, melhorar 
a precisão e aumentar a capacidade de evasão de obstáculos.

Recentemente,~\cite{li2023pseudo} propuseram um novo esquema de \emph{Obstacle avoidance} (OA) para manipuladores redundantes 
baseado em na pseudo-inversa da matriz jacobiana no nível das acelerações das juntas, demonstrando a viabilidade do método através de simulações 
com o manipulador robótico PA10 o qual possui \(7\) DOFs. Por outro lado,~\cite{kuri2023som}
abordaram uma técnica de controle inteligente para um manipulador robótico espacial com 5 DOFs para executar
trajetórias tridimensionais desejadas (linha, circular e elíptica) enquanto resolve a redundância 
(evitação dos limites das juntas) mostrando que a depender da tarefa, mesmo para manipuladores com um número menor de DOFs, 
a redundância pode ser explorada para melhorar o desempenho do robô.

O trabalho de \cite{ancona2017} validou técnicas de resolução de redundância em diversos cenários, destacando a importância da escolha do critério de desempenho
para direcionar o comportamento do manipulador. A metodologia proposta, baseada em um procedimento de cinemática inversa e um controlador orientado a objetos,
mostrou-se eficaz e versátil tanto em simulações quanto na prática real. Além disso, o trabalho explorou a manipulação móvel redundante para operações hábeis 
e interação segura entre humanos e robôs em ambientes industriais, com uma contribuição importante na arquitetura de controle de software proposta, que é escalável,
portátil, integrável com tecnologias heterogêneas e pronta para uso em ambientes de produção.

Por fim, \cite{hammond2011} propôs um método heurístico para configurar manipuladores robóticos com redundância cinemática, visando aumentar a precisão 
dos movimentos em tarefas específicas, utilizando o índice de resolução de movimento efetivo (EMR). Os resultados mostram que o índice EMR pode ser 
usado tanto como uma métrica de aptidão em otimização de design quanto como um objetivo na resolução de redundância,  destacando o potencial dessas
abordagens em aplicações de micromanipulação e montagem de precisão,  melhorando significativamente a resolução de movimento em trajetórias específicas 
sem a necessidade de componentes de alta precisão e custo elevado.

\section{Motivação}\label{sec:motivation}

A abordagem tradicional no projeto de manipuladores teve como foco principal a minimização de custos e manutenção,
utilizando o número mínimo de juntas necessárias para uma executar uma determinada tarefa levando, por exemplo, ao
desenvolvimento dos robôs \emph{Selective Compliance Assembly Robot Arm} (SCARA) para operações de \emph{pick-and-place}~\cite{siciliano_springer_2008}.
No entanto, essa abordagem  minimalista apresenta uma série de limitações em aplicações do mundo real, onde fatores como
limites de juntas, singularidades e obstáculos no espaço de trabalho estão presentes. Ao ter mais DOFs
do que o estritamente necessário, os manipuladores redundantes podem alcançar maior destreza e versatilidade, tornando-os
mais adequados para ambientes complexos e dinamicamente mutáveis.

Para além das restrições cinemáticas, a redundância permite a otimização de métricas de desempenho, como a minimização do torque ou do consumo de energia, melhorando a
eficiência geral do sistema. Vale ressaltar que projetar manipuladores com juntas adicionais e garantir sua confiabilidade operacional é um
processo complexo e custoso. Esquemas eficazes de resolução de redundância são críticos para o sucesso do planejamento e controle de movimentos,
especialmente em ambientes dinâmicos. Apesar desses desafios, os benefícios da redundância cinemática em aumentar a destreza, versatilidade e eficiência
fazem dela uma abordagem interessante em sistemas robóticos avançados.

\section{Objetivos}\label{sec:objectives}

\subsection{Objetivos Gerais}

Estudar e implementar a modelagem de cinemática inversa diferencial em manipuladores seriais redundantes, possibilitando a execução de
trajetórias no espaço de trabalho que levam em conta não só as restrições cinemáticas do movimento mas também critérios
de desempenho do manipulador.

\subsection{Objetivos Específicos}
\begin{itemize}
	\item Estudar a modelagem de cadeias cinemáticas em manipuladores robóticos seriais
	\item Investigar o uso da cinemática direta e inversa diferenciais para execução de trajetórias cartesianas
	\item Implementar a lei de controle utilizando a cinemática direta diferencial e esquemas de resolução de redundância utilizando a pseudo inversa da matriz jacobiana
	\item Propor um ambiente virtual para execução de simulações utilizando o modelo de um manipulador robótico com 5 graus de liberdade
	\item Analisar o desempenho do esquema de controle em diferentes cenários de execução de trajetórias e critérios de desempenho.
\end{itemize}

\section{Estrutura do texto}\label{sec:structure}

Este trabalho está estruturado de modo a introduzir os conceitos já difundidos da cinemática diferencial em manipuladores redundantes,
bem como apresentar uma abordagem prática para resolução de redundância de um manipulador planar simples, com cinco graus de liberdade.
No capítulo 2, é apresentada a fundamentação teórica, abordando conceitos essenciais como a representação de poses no
espaço, transformações homogêneas, cinemática direta, cinemática diferencial, e a resolução de redundância.
O capítulo 3 descreve a implementação prática, detalhando a simulação de manipuladores robóticos, a arquitetura de comunicação proposta
e o algoritmo \emph{Resolved Rate Motion Control}. No capítulo 4, são apresentados os resultados experimentais divididos em dois cenários
distintos, seguidos pela conclusão no capítulo 5, onde são discutidos os principais pontos abordados e sugestões para trabalhos futuros.
