\chapter*{Abstract}
\noindent Kinematically redundant manipulators have more degrees of freedom (DOFs) than necessary to perform a specific task,
 offering greater flexibility and dexterity in executing trajectories. However, these additional DOFs also introduce increased
  complexity in the manipulator's inverse kinematics, necessitating redundancy resolution schemes to find a particular solution 
  that optimizes performance criteria such as joint speed, manipulability, or distance to obstacles. In this work, we present a 
  theoretical foundation on the key aspects of differential inverse kinematics for redundant manipulators. Subsequently, we 
  implemented a simulated environment using the Webots simulator and the Robot Operating System (ROS), allowing practical 
  application of the \emph{Resolved Rate Motion Control} (RRMC) scheme. Our experiments demonstrated that the controller can 
  optimize performance metrics without violating primary kinematic constraints in different scenarios of Cartesian trajectory execution.

  \vspace{5mm}

\noindent\textbf{
    \textit{Keywords}:~\textit{robotic manipulators};~\textit{differential inverse kinematics};
    ~\textit{redundancy resolution};~\textit{robotics simulation};~\textit{resolved rate motion control}.
}
