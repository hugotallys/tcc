\chapter*{Abstract}
\noindent Kinematically redundant manipulators have more degrees of freedom (DoFs) 
than necessary to perform a specific task, offering greater flexibility and dexterity 
in executing trajectories. However, these additional DoFs also introduce greater complexity 
in the manipulator's inverse kinematics, requiring the use of redundancy resolution schemes 
to find a particular solution that optimizes the manipulator's performance criteria, such as 
joint speed, manipulability, or distance to obstacles. This work presents a theoretical foundation
 on the main points related to differential inverse kinematics for redundant manipulators. To 
 validate the proposed methodology, a virtual environment was implemented using the Webots simulator 
 and the Robot Operating System (ROS), allowing the practical application of the \emph{Resolved Rate Motion Control} 
 (RRMC) control scheme. The simulations carried out show that the controller is capable of optimizing 
 performance metrics without violating the primary kinematic constraints in different Cartesian trajectory 
 execution scenarios.

  \vspace{5mm}

\noindent\textbf{
    \textit{Keywords}:~\textit{robotic manipulators};~\textit{differential inverse kinematics};
    ~\textit{redundancy resolution};~\textit{robotics simulation};~\textit{resolved rate motion control}.
}
