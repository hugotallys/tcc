\chapter*{Resumo}
\noindent Manipuladores cinematicamente redundantes possuem mais graus de liberdade (DOFs) do 
que o necessário para realizar uma tarefa específica, oferecendo maior flexibilidade e destreza na execução de trajetórias. 
No entanto, esses DOFs adicionais também introduzem uma maior complexidade na cinemática inversa do manipulador, sendo necessário 
recorrer a esquemas de resolução de redundância, para encontrar uma solução particular otimizando critérios de desempenho do manipulador,
como velocidade das juntas, manipulabilidade ou distância para obstáculos. Neste trabalho, apresentamos uma fundamentação teórica sobre 
os principais pontos relativos a cinemática inversa diferencial voltada para manipuladores redundantes. Em seguida, implementamos
um ambiente simulado utilizando o simulador Webots e o Sistema Operacional de Robôs (ROS), permitindo a aplicação prática do 
esquema de controle \emph{Resolved Rate Motion Control} (RRMC). Nossos experimentos mostraram que o controlador é capaz de otimizar
métricas de desempenho, sem violar as restrições cinemáticas primárias em diferentes cenários de execução de trajetórias cartesianas.

\vspace{5mm}

\noindent\textbf{
    \textit{Palavras-chave}:~\textit{manipuladores robóticos};~\textit{cinemática inversa diferencial};
    ~\textit{resolução de redundância};~\textit{simulação de robótica};~\textit{resolved rate motion control}.
}
