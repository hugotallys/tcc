\chapter{Conclusão}\label{cap:conclusao}

Este trabalho discutiu a aplicação de esquemas de controle com resolução de 
redundância para abordar os desafios impostos pelas singularidades na cinemática 
diferencial inversa de um manipulador serial com cinco graus de liberdade. Foi proposto ambiente completamente
simulado para testar e validar a estratégia de controle, proporcionando também uma comunicação robusta 
entre controlador e o modelo virtual do manipulador usando o Sistema Operacional de Robôs (ROS). 
A abordagem foi avaliada em cenários distintos, propiciando uma análise qualitativa da execução de
trajetórias em diferentes condições de restrições cinemáticas e de desempenho do manipulador.

Perspectivas futuras para o trabalho, incluem o estudo da escolha ótima de um ganho \(k_0\), visando otimizar 
o desempenho do manipulador em diferentes cenários e também uma análise teórica/estatística mais aprofundada dos resultados obtidos, com o objetivo de avaliar a robustez do
controlador e a convergência para soluções ótimas. Além disso, é desejável explorar cadeias cinemáticas 
mais complexas, como as dos manipuladores do tipo \emph{elbow} e \emph{wrist}, que possuem um maior 
grau de redundância tipicamente maior que 7, permitindo a resolução não so a nível do controle da posição, mas 
também da orientação do efetuador final. Vale ressaltar também que uma continuidade natural para o trabalho 
é a transferência dos resultados para um robô real e a integração em um \emph{framework} de planejamento de trajetórias, permitindo
a validação do esquema de controle em cenários mais realistas e desafiadores.